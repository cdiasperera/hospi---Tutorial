Deadlocks, in concurrent systems, is a state of a concurrent system where no process can make progress as it is waiting upon another resource. Since all resources are waiting, no progress is made in the system. A (live)lock is a state where the system is evolving but making no ``progress". Intuitively, there is some circular state change preventing some other aspect of the process from continuing.

Naturally, (dead)locks in a system are undesirable and there is a rich theory \todo{Link theory} surrounding detecting, preventing and avoiding (dead)locks.

Fantastically, the session typed (higher-order) the $\pi$-calculus actually does provide functions that, given some process, it can detect whether this process will actually (dead)lock. 

% Again, this theory is not strictly necessary to using the implementation, and interested readers can find out more about it in appendix \ref{deadlocktheory}