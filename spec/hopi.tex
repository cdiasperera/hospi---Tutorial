To define session channels (say $s$), we write \code{s\{0\}}. To define a shared channel, (say $a$) we write \code{a(0)} (observe the use of braces vs curved parentheses).

To define the dual of a session / shared channel, we write: \code{*(x\{0\})} / \code{*(x(0))}.

The following tables lists the process / type syntax for hos$\pi$ calculus and the corresponding syntax (with the (pseudo)-signature) in Maude:


\newcommand{\tableline}[3]{#1 & #2 & \code{#3} \\ \hline}

\begin{table}[H]
\centering
\begin{tabular}{|c|c|c|}
\hline
Non-Terminal & HOS$\pi$ & Maude Signature  \\ \hline
\tableline{$V$}{$\lambda x . P$}{op lambda('x') . P : Qid Trm -> Value}
\tableline{$V$}{$\lambda (x : S) . P$}{op lambda('x' : S) . P : Qid Type -> Value}
\tableline{$p$}{$x<v>$}{op 'x'{0}< v > : Chan Value -> Guard}
\tableline{$p$}{$x(v)$}{op 'x'{0}('z') : Chan Qid -> Guard}
\tableline{$P$}{$\ep$}{op : -> Trm}
\tableline{$P$}{$(\rest x) P$}{op new['x'] P : Qid Trm -> Trm}
\tableline{$P$}{$(\rest x : S) P$}{op new['x' : S] P : Qid Trm -> Trm}
\tableline{$P$}{$L x$}{op 'L'{0}{'x'{0}} : Value Chan -> Trm}
\tableline{$P$}{$X$}{op rec(X) : Qid -> Trm}
\tableline{$P$}{$\mu X . P$}{op u['X'] . P : Qid -> Trm}
\tableline{$P$}{$\mu X : S . P$}{op u['X' : S] . P : Qid Type Trm -> Trm}
\tableline{$P$}{$P \mid Q$}{op P | Q : Trm Trm -> Trm}
\tableline{$P$}{$x << l . P$}{op 'x'{0} << l . P : Chan Qid Trm -> Trm}
\tableline{$P$}{$x >> \set{C}$}{op 'x'{0} >> {C} : Chan Choiceset -> Trm}
\tableline{$P$}{$p . P$}{op p . P : Guard Trm -> Trm}
\tableline{$c$}{$l : P$}{op l : P : Qid Trm -> Choice}
\tableline{$C$}{$C1 \cup C2$}{C1; C2 : Choiceset Choiceset -> Choiceset}
\tableline{$C$}{$\emptyset$}{empty}
\end{tabular}
\end{table}

Note that for the guards and branching ($p$), we have used session channels (\code{'x'{0}}) though it would've been equally valid to have used shared channels (\code{'x'(0)}).

Note how recursive variables, restrictions and abstractions have additional syntax to define their bound variables type. This is used for type-checking. 

For abstractions, when defining the process \code{P} in \code{lambda('x') . P}, to use \code{'x'} in \code{P}, we write \code{'x'{0}}.

Note that \code{Choice < Choiceset}. That is a to say, a \code{choice} is also a \code{Choiceset}. Formally, $x \in \set{y \mid y \text{ is a } c} \implies x \in \set{y \mid y \text{ is a } C}$.

\begin{table}[H]
\centering
\begin{tabular}{|c|c|c|}
\hline
Non-Terminal & $\pi$-calculus & Maude Signature  \\ \hline
\tableline{T}{$? T_1 . T_2$}{op ? T1 . T2 : Type SessionType -> SessionType}
\tableline{T}{$! T_1 . T_2$}{op ! T1 . T2 : Type SessionType -> SessionType} 
\tableline{T}{$\oplus{CT}$}{op +{CT} : ChoiceTSet -> SessionType}
\tableline{T}{$\&{CT}$}{op \&{CT} : ChoiceTSet -> SessionType}
\tableline{T}{$\mu X. P$}{op u['X'] . P : Qid SessionType -> SessionType}
\tableline{T}{$X$}{op var('X') : Qid -> SessionType}
\tableline{T}{$\et$}{op end : -> SessionType}
\tableline{T}{$\lho{C}$}{op lin proc(C, <>) : Type -> LinearType}
\tableline{T}{$\sho{C}$}{op un proc(C, <>) : Type -> SharedType}
\tableline{T}{$shared{T}$}{op [T] : Type -> SharedType}
\tableline{T}{$\diamond$}{op <> : -> Type}
\tableline{ct}{$l : T$}{op l : T : Qid SessionType -> ChoiceT}
\tableline{CT}{$C_1 \cup C_2$}{op C1 ; C2 : ChoiceTSet ChoiceTSet -> ChoiceTSet}
\tableline{CT}{$\emptyset$}{op empty : -> ChoiceTSet}

\end{tabular}
\end{table}

Note that \code{ChoiceT < ChoiceTset}. That is a to say, a \code{choiceT} is also a \code{ChoiceTset}. Formally, $x \in \set{y \mid y \text{ is a } ct} \implies x \in \set{y \mid y \text{ is a } CT}$.

\todo{captions , refering of tables}