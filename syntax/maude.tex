To install Maude, please follow the instructions at \href{http://maude.cs.illinois.edu/w/index.php/Maude_download_and_installation}{this page}.
Maude is a language as deep and broad as the pacific ocean, but you only need to know a very limited syntax to be able to actually use it for the purposes of using the specification.

There are three broad parts to a Maude program (for our purposes):
\begin{enumerate}
    \item The loading of the required modules
    \item The definition of various entities 
    \item The execution of various functions on those entities (or inline defined entities)
\end{enumerate}

To load modules, we need the \code{load} command at the start of our file. Furthermore, you will need to use a \code{protecting} declaration on certain modules to avoid ``junk" being added (you will see what modules need to be protected in section \ref{using}).

To define entities, you have to define a formal \textbf{module}. To do this, we use the following syntax:

\begin{verbatim}
mod MODULE_NAME is

endm
\end{verbatim}
\todo{improve listing environment}
With the definition of entities existing within these two lines.

In practice, your files dedicated to using the specification will look something like this (note the protecting declarations, which only apply ):
\begin{verbatim}
load chan
load cinni
load chanset
load hospi-syntax
load hospi-semantics
load hospi-types-syntax
load hospi-types-funcs
load hospi-types

mod TEST is
  protecting HOSPI-SYNTAX .
  protecting HOSPI-SEMANTICS .
  protecting HOSPI-TYPES .

--- <Entity definitions go here>
endm

<Function declarations go here>

\end{verbatim}

Now, we shall discuss how you define entities / use functions. Note that every statement in Maude that we will discuss are all appended with ``\code{.}" (a dot), including protecting declarations. In some sense, this is Maude's equivalent of a semicolon in C, except that the dot can \textit{also} be used in your definition of Maude's equivalent of functions.

To define a function in Maude, there are two parts:
\begin{itemize}
    \item Defining the signature of it
    \item Defining it's semantics in a functional style.
\end{itemize}

To define the signature we have the following (pseudo-)syntax\footnote{Note that this syntax isn't exactly mapping to that in Maude. The dots here are meant to represent a continuation of the space separated list of parameters to the $N$\textsuperscript{th} parameters}:

\begin{verbatim}
op functionName : Param1 Param2 ...  ParamN -> Result .
\end{verbatim}

where \verb`Param1`, \verb`Param2`, \verb`ParamN`, \verb`Result` are types (called \textbf{sorts} in Maude). Note that it's possible to have $N = 0$ parameters.

Then, this function name can be applied in the usual manner:
\begin{verbatim}
functionName(arg1, arg2, ..., argN) .
\end{verbatim}

where \code{arg1} is of sort \code{argN}.

It's also possible to encode the function name in-line, using \code{\_}:

\begin{verbatim}
op _ functionName _ : Param1 Param2 -> Result .
\end{verbatim}

In general, one would need $N$ underscores for $N$ params, and it does not all need to be infix, nor does it require the same ``name" to be use. For example, the following would be valid:

\begin{verbatim}
op __ f __ g _ : Param1 Param2 Param3 Param4 Param5 -> Result .
\end{verbatim}

Then, to define the semantics of a these operators, we have to define equalities with the \code{eq} code word:

\begin{verbatim}
eq a functionName b = c
\end{verbatim}

Those with experience in Haskell will find this syntax familiar.

This is the underpinning of Maude. Using these equations, it can transform existing statements into other statements, which is the core of our $\pi$-calculus. There is, of course, much much more to defining these theories (such as attaching properties to them) but it's not important to know, for now.

For the functions, there are two commands we need to execute them.

The \code{red} (short for \code{reduce}) command performs rewriting on equations, which we will use to observe the evolution of our systme.

The \code{rew} (short for \code{rewrite}) command allows us to more systematically explore the reduction. \todo{Ask: What's the difference actually? When to use rew vs red?}

\code{modelCheck} is a function that will be used to check for deadlocks, though in a very specific manner so it's not particularly important that you understand the syntax deeply.

We will primarily see lines in the form:

\begin{verbatim}
red functionName(a,b) .
\end{verbatim}

Lastly, one-line comments are written by prefixing a line with \code{---} and multiline comments are written as such:
\begin{verbatim}
***(
    Comments go here
)
\end{verbatim}

If you wish to dive deeper into the syntax (and theory!) of Maude, you can read the textbook \cite{maude} (that also acts as a specification) on it. Their \href{http://maude.cs.illinois.edu/}{website} contains a free-pdf.