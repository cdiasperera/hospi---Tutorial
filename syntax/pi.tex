\todo{Ask Prof: I guess this is technically a session based $\pi$-calc. Worth specifying?}
\subsubsection{Process Syntax}
A process (which is ranged over by $P,Q,R$), in the $\pi$-calculus, is defined recursively. With the fundamental operation being the ability to output values and to take in inputs, in some manner. To this end, we need to define a very what values can actually be passed. For the specification, we restrict ourselves to boolean values $\verb`true`, \verb`false`$, and of course, channel names themselves, which we will use the variables $x,y,z$ to range over.

You might consider this quite limiting. Perhaps you'd like to pass around object, a la an Object-Oriented Programming paradigm, and call methods upon that object. Alas, this specification does not extend that ability, but there are extensions that be applied to the $\pi$-calculus to allow this facility, see \href{https://www.di.fc.ul.pt/~vv/papers/vasconcelos_session-types-programming.pdf }{here}.

\sss{Values}
Let us now proceed to define the syntax of our $\pi$-caluclus. If we let $v$ range over all possible values, we have:

\begin{align*}
    v ::= x \mid \text{true} \mid \text{false}
\end{align*}

\sss{Full process syntax}
The full syntax for a process is as follows:

\begin{align*}
q &::= \un \mid \lin \\
P &::= \ep \mid (\rest xy) P \mid \bar{x}v.P \mid q x(y). P \mid (P_1 \mid P_2) \mid \ifelsethen{v}{P_1}{P_2} \mid  x \offer \set{l_i : P_i}_{i \in I} \mid x \select l . P
\end{align*}

This might be quite overwhelming, but we will analyze this syntax by studying each form a process can take.

Note that when discussing the interpretations of these terms, the syntax is not formally defining such interpretations. The operational semantics will more formally define these concepts. 

\sss{End Process}
As the $\pi$-calculus is defined recursively, so we will need some ``base" process, which we denote as $\ep$.

Thus, we might have a process $P$ where:
\begin{align*}
    P = \ep
\end{align*}

\sss{Parallelization}

Now, the key operator in the $\pi$-calculus is the parallelization operator. In the $\pi$-calculus, we actually define two processes that are in parallel to be a process itself!

\begin{align*}
    P = P_1 \mid P_2
\end{align*}

In some sense, this is the ``core" of the $\pi$-calculus.

\sss{Restriction}
Next, we want some means of actually transferring values. Let us imagine a channel double-ended conduit, which processes can speak/listen at each end. To ``declare" such a line, we have:

\begin{align*}
P = (\rest xy)P_1
\end{align*}

Which says that process $P$ can use endpoints $x,y$ to communicate between sub-processes in $P_1$.

This declares that the resulting process $P$ can use $x,y$ are opposite endpoints of the same channel. You might wonder the use of this, since isn't the purpose of endpoints so that processes can communicate? Keep reading, it will be clearer as we move forward.

Also note that in this process, we declare $x,y$ to be ``new" with respect to any other channel point everywhere. That is, we can only ``correctly" use $\rest$ upon $x,y$ if $x,y$ is used no where else in the process, either internally, or in processes that are in parallel with it. This is actually only enforced via the typing rules.

This operation is known as a ``restriction", since we are restricting the use of these channel names to the process $P_1$.

\sss{Output}
Suppose
\begin{align*}
    P = \bar{x}v.P_1
\end{align*}

This means that $P$ is a process outputs the value $v$ on the channel $x$, and then continues as the \textbf{continuation process} $P_1$. Note that in a well-defined process, by the time that the action is ``evaluated", $x$ will actually be a channel itself.

\sss{Input}
Suppose
\begin{align*}
    P = q x(y).P_1
\end{align*}

The above statement means that $P$ is a process which takes in input upon from endpoint $x$ and any instance of $y$ in $P$ is replaced with the value we picked up.

Now, generally speaking, we might wish for our process to continue indefinitely (say, if it's representing a server for an API). This is the purpose of the $q$ qualifier in the process $P$. To this end, we introduce the $\un$ and $\lin$ qualifiers:

\begin{align*}
    q = \un \mid \lin
\end{align*}

which prefix our input types (thus resulting in a slight modification for our input definition):

An input action prefixed by $\lin$ corresponds to our current understanding of input actions, meaning that once the input is taken, the continuation process $P_1$ runs with the corresponding substitution.

However, if the input action is prefixed by $\un$, then this \textit{also} implies a \textbf{new} process spawns that is exactly the same as the process $P$ (before the input arrived). Or equivalently (and this is the interpretation we will be proceeding with), we spawn a process $P[v/y]$ (where $v$ is the value being inputted), and the process continues, acting as if nothing happened.

\sss{Conditional Processes}
Thus far, we haven't introduced any syntax that takes advantage of the boolean value. To introduce the familiar concept of conditional statements, we have the following syntax:

\begin{align*}
    P = \ifelsethen{v}{P_1}{P_2}
\end{align*}

This is a process where if the received value $v$ is $\text{true}$ then $P_1$ is run, otherwise $P_2$ is run.

\sss{Branching}
Lastly, suppose we wished to offer a selection of possible processes that might run, after some input. This somewhat can be done with our conditional processes mentioned earlier, with a logarithmically scaling amount of input actions required before a choice can be made. The $\pi$-calculus offers concise notation:

\begin{align*}
    P &= x \offer \set{l_i : P_i}_{i \in I} \\
    P &= x \select l.P 
\end{align*}

The top process indicate that it expects another process to send through channel $x$ a label from the set $I$, upon which it will run the corresponding process. The bottom process indicates that the process will send label $l$ through channel x.

\subsubsection{Operational Semantics}

When discussing the syntax for processes above, we gave some intuitions for what that syntax meant, in terms of how it interacted with other processes, i.e: how that term contributes to overall process evolution. Broadly, this is known as the \textbf{operational semantics} of the $\pi$-calculus, and we will now more formally describe these.

These are written as reductions with the $\ra$ relation. Informally, you can interpret $P \ra P'$ to mean that $P$ becomes $P'$ after one step of the evolution (i.e: a \textit{reduction}).

The first rule we introduce involve value passing into an input action with the $\lin$ qualifier.:

$$(\rest xy)(\bar{x}v.P \mid \lin y(z).Q \mid R) \ra (\rest xy)(P \mid Q[v/z]\mid R)$$

This rule dictates that if from process $\bar{x}v.P$ we are outputting value $v$ from endpoint $x$, and if there is a corresponding endpoint $y$ for some input action for process $\lin y(z).Q$, then the former process sends $v$ along that channel and evolves into $P$, with the latter process evolving into $Q$ but replacing all instaces of $z$ in $Q$ with $v$. As you can see, other processes (encapsulated by $R$) proceed unaffected.

A similar rule involves value passing into an input action with the $\un$ qualifier:

$$(\rest xy)(\bar{x}v.P \mid \un y(z).Q \mid R) \ra (\rest xy)(P \mid Q[v/z]\mid \un y(z).Q \mid R)$$

This rule dictates the same semantics as the previous rule, with the exception that using $\un y(z).Q$ replicates it.

The last rule we present for communication involves communication via labels.:

$$\frac{j \in I}{(\rest xy)(x \select l_j . P \mid y \offer \set{l_i : Q_i}_{i \in I} \mid R) \ra (\rest xy)(P \mid Q_j \mid R)}$$

This rule dictates that if you have a process that offers various processes along endpoint $y$, if another process is sending an applicable label along the corresponding endpoint $x$, then the former process evolves into the selected process, and the latter process proceeds with whatever other work it is going to do.

One interesting observation you might have made at this point is that all these rules occur under a restriction ($\rest$) of the endpoints for the channel being used to pass values / labels. Indeed, this suggests that well-formed $\pi$-calculus processes can only communicate with another endpoint if those endpoints are first related as being corresponding endpoints of a channel.

The following rules relate to ``internal" evolution, where if a process can evolve without outside interference, it can also do so, if it is in parallel with another operation, or if it under restriction of some channel:

\begin{gather*}
\frac{P \ra P'}{P \mid Q \ra P' \mid Q} \\
\frac{P \ra P'}{(\rest xy)P \ra (\rest xy) P')}
\end{gather*}

You might be curious why the former rule doesn't have an equivalent rule, for $Q \ra Q'$. We will see that \textbf{structural congruence} (defined below) covers this implicitly.

The last rule we present has to do with structural congruence explicitly, which says that ``equivalent" processes can follow the same reduction steps. This corresponds to our intuition of equivalence, since processes that we deem equivalent should be able to evolve in the same manner (if it cannot, it begs the question of why we consider them equivalent):

$$
\frac{P \equiv P' \quad P' \ra Q' \quad Q \equiv Q'}{P \ra Q}
$$

\subsubsection{Structural Congruence}
As mentioned earlier, we discussed this notion of ``equivalent processes". In some sense, these are processes that ``should" follow the same behavior, if we define our syntax correctly. We define equivalent processes via the equivalence relation $\equiv$.

Firstly, we observe that since the $\ep$ does ``nothing", putting it into parallel with another process (or defining some channel upon it) does nothing:

\begin{align*}
P \mid \ep &\equiv P \\
(\rest xy) \ep &\equiv \ep
\end{align*}

Next, we wish to define our parallel composition operator to be both commutative and associative:
\begin{align*}
P \mid Q &\equiv Q \mid P \\
P \mid (Q \mid R) &\equiv (P \mid Q) \mid R
\end{align*}

Similar to the commutative rule, observe that it doesn't matter in which order we declare our channel endpoints:

$$
(\rest xy)(\rest wz)P \equiv (\rest wz)(\rest xy)P
$$

Next we define how how $\ifelsethen{}{}{}$ actually operates:
\todo{Ask prof: Why don't we define the true/false struct cong as op sems?}

\begin{align*}
\ifelsethen{\text{true}}{P_1}{P_2} &\equiv P_1 \\
\ifelsethen{\text{false}}{P_1}{P_2} &\equiv P_2
\end{align*}

You might wonder if this couldn't have been defined as part of the operational semantics. It potentially could've, but note that defining something to be structurally congruent is actually a stronger claim, and thus, it is defined in this manner.

Lastly, suppose we have some process $(\rest xy)P$. This means that $x,y$ is only used within $P$. If you parallely compose it with $Q$ which does not ``outwardly" use $x,y$, it means you can expand the scope of the restriction over it:


What do we mean by ``outwardly" compose? This means that you could, in principle, rename the $x,y$ in $Q$ without changing the behavior of $Q$. In a formal sense, we say that these $x,y$ in $Q$ are ``bound" to $Q$. All such $x,y$ in $Q$ are in the set $bv(Q)$. Any other identifiers in $Q$ that are not in $bv(Q)$ are in $fv(Q)$. So we formally express this intuition as:

$$
x,y \notin fv(Q) \implies (\rest xy)P \mid Q \equiv (\rest xy)(P \mid Q)
$$

\todo{Probs put this in the appendix, check with Prof about labels}

So which values are in $fv(Q)$? Formally, $fv(Q)$ is defined as the set of variables that are not bound in $Q$. A bound variable is a variable that is binded by the input or restriction actions. That is in:
\begin{gather*}
Q = qx(y).P \\
R = (\rest y_1, y_2) P
\end{gather*}

The input action in $Q$ binds $y$ and the restriction action in $R$ binds $y_1, y_2$.

It might be worth exploring this definition in practice. Say you wished to determine the free variables in $Q$. We can perform this inductively, upon the structure of a process:

\begin{itemize}
    \item $Q = \ep$: No free variables are added
    \item $Q = \bar{x}v.P$: $fv(P), x,v$ are added to the list of free variables
    \item $Q = q x(y).P$: $fv(P), x$ is added to the list of free variables but $y$ is \textbf{not}. This is because if we wanted to use $y$ at some other restriction, we can simply rename $y$ and all instance of $y$ in $P$.
    \item $Q = P_1 \mid P_2$: $fv(P_1), fv(P_2)$ are added to the list of free variables
    \item $Q = (\rest xy)P$: $fv(P)$ is added to the list of free variables but \textbf{not} $x,y$. Again, if we wanted to use $x,y$ in some other restriction, we can rename all instances of it in $P$ as well to something else.
    \item $Q = \ifelsethen{v}{P_1}{P_2}$: $v,fv(P_1), fv(P_2)$ are added to the list of free variables.
    \item $Q = x \offer \set{l_i : P_i}_{i \in I}$: $x$ and $\forall i \in I: fv(P_i), l_i$ are added to the list of free variables.
    \item $Q = x \select l.P$: $x, fv(P), l$ are added to the list of free variables.
\end{itemize}

\subsubsection{Type Syntax}
As mentioned earlier, our syntax surrounding the $\pi$-calculus is too loose, in that they don't necessarily conform to our intuition behind their usage, or you can create processes that do nothing / don't make sense. To resolve this, we turn to typing.

Now, we do not actually type processes, since processes aren't true values of calculus. Rather, what we are (mainly) typing is the ``behavior" of channel endpoints themselves. That is, if we have some process, its free names must correspond to some protocol (as specified by the type of that endpoint). We consider this information (regarding channel typing) the ``context" of our process.

Making sure that our processes satisfy its context is a recursive process, as processes are built on top of processes. As we delve deeper into a process to ensure it's satisfaction of a type, the context required for the evaluation will also change, and these modified contexts are what allows us to ensure the internal consistency of our processes. As a brief concrete example, evaluating the a process under restriction will bring the channel under restriction into the context, thus evolving our context.This explanation might be too abstract, which is fine. It becomes clearer once you see the actual rules and some examples.

Before we turn to typing, we will introduce one one concept upon channels. Recall we mentioned that free variables in a process need to correspond to some protocol. Some protocols require that a channel endpoint can only be used by \textbf{one} process. That is, it cannot jump between processes. To this end, we specify such endpoints with $\lin$, and other endpoints with $\un$.

Note that it's important to decouple the meanings of $\lin, \un$ with respect to typing channel endpoints and the effects of $\lin, \un$ with respect to its application on input channels. This is an unfortunate \todo{Ask Prof: Feels unfortunate to me?} matter of notation.

Now, before we define these typing rules, we need some means of actually \textit{writing} a type, which we now present.

\sss{Full Typing Syntax}
Below is the full syntax for types:

\begin{align*}
p &::=\ !T_1.T_2 \mid ?T_1.T_2 \mid \&\set{l_i : T_i}_{i \in I} \mid \oplus \set{l_I : T_i}_{i \in I} \\
T &::=\ \et \mid \bool \mid q p \mid \mu a . T \mid
\end{align*}

As with our process syntax, we will now explore these constructs indivdually

\sss{Base Cases}
We define a type $T$ recursively, with a base case being the type of channel endpoint that does nothing (i.e: $\ep$) and a Boolean value, respectively:

\begin{gather*}
T_1 = \et \\
T_2 = \bool
\end{gather*}

\sss{Types for channel operations}

Now, suppose a process involves either:
\begin{itemize}
    \item Restriction
    \item Parallelization
    \item A conditional choice
\end{itemize}

If you consider it, these operations do not actually involve the usage of a channel, and thus, it's affect on a channel is not reflected in its type. Of course, restriction does place some additional restraints on a channel, but these restraints will be reflected in the typing rules. Thus, these operations do not have any explicit terms associated with it!

However, the remaining operators (input / output / branching / selection ) \textit{do} involve the usage of a channel, and they have the following syntactic elements:

\begin{align*}
p_1 &=\ !T_1.T_2 \\
p_2 &=\ ?T_1.T_2 \\
p_3 &=\ \&\set{l_i : T_i}_{i \in I} \\
p_4 &=\ \oplus \set{l_i : T_i}_{i \in I} \\
 \end{align*}

If we have some statement $x : q p_j$, then with:
\begin{itemize}
    \item $j = 1$: This channel outputs a value with type $T_1$ and then channel $x$ acts according to $T_2$
    \item $j = 2$: This channel takes in an input with type $T_1$ and then channel $x$ acts according to $T_2$
    \item $j = 3$: This channel takes in some label $l_i$, and then acts as $T_i$, for each $i \in I$
    \item $j = 4$: This channel outputs some label, and then acts as the continuation type $T_i$, if it sent $l_i$.
\end{itemize}

If you read the last two types carelessly, you might be led to believe that a label has a type, but $T_i$ reflects the \textit{continuation type} of a channel, after the corresponding action has been taken. Furthermore, for a selection type ($\oplus$), the type $T_i$ is the continuation type for the process that is doing the selection, and is not the continuation type for the other process that responds to the selection.

Observe that we define $p$, but not additional elements of $T$. This is because all these operations act on a channel, which can be linear  or unrestricted. In this sense, it is a \textbf{pretype}.

To use it, we have to combine it with a the linear ($\lin$) or unrestricted ($\un$) \textbf{qualifier}. Hence, we might have a type:

$$
T = q\ p
$$

\sss{Infinite Behavior / Replication}

Lastly, one might wonder how we type a process that is replicating (that is, has a $\un$ prefix). To do this, we essentially explicitly type it as such:

$$
T = \mu a.T_1
$$

This is relatively novel syntax, but essentially, it defines $a$ to be a type variable that may appear in $T_1$, that represents $T_1$ itself. So, for example

$$
\mu a. \un ?\bool . a
$$

is equivalent to

$$
\mu a . \un ?\ \bool .\ \un ?\ \bool . a
$$

showing that we can endlessly replicate the $T$ part of $\mu a . T$, hence typing infinite behavior.

Lastly, recall we spoke of a context. We formalize this notion in $\Gamma$, where $\Gamma$ is the context under which our typing rules are evaluated:

$$
\Gamma ::= \emptyset \mid \Gamma, x : T
$$

This reads that the base context is the empty set, and we can add to this context via appending the type of a value $x$.

And thus, this defines our typing syntax!

\subsubsection{Typing Rules}

\textbf{Note: This subsection discusses and presents how typing rules actually enforce all the intuitions / assertions of the behavior of syntactic elements we've since discussed. It's not essential to read, but reading this section will probably fix any incorrect understanding of those intuitions. A more practical reader would probably skip this section and come back when a type doesn't seem to work as expected.}

\sss{Typing rules for values}
First, we will type actual values, such as channels and boolean values. Recall that processes aren't values themselves, in the sense you cannot pass processes through channels. Such typing rules will have form:

$$
\Gamma_1 \vdash v : T ; \Gamma_2
$$

where $v$ is either a channel variable or a boolean variable. This reads as: $v$ is typed as $T$, where with a ``remaining" context $\Gamma_2$. This principle here is that $\Gamma_2$ contains the context that the ``next" typing rule must be evaluated against. It's main purpose is to ensure if $x$ is a linear channel, $\Gamma_2$ does not contain a context that has $x$.

The rules are as follows:
\todo{Explain Rules}

\begin{gather*}
\frac{}{\Gamma \vdash \true : \bool ; \Gamma} \\
\frac{}{\Gamma \vdash \false : \bool ; \Gamma} \\
\frac{}{\Gamma_1, x : \lin p , \Gamma_2 \vdash x : \lin p; (\Gamma_1, \Gamma_2)} \\
\frac{\unf{T}\footnote{This makes sure that $T$ is defined along a unrestricted channel or is a boolean value.}}{\Gamma_1, x : T , \Gamma_2 \vdash x : T; (\Gamma_1, x : T, \Gamma_2)} \\
\end{gather*}

Note that $\unf{T}$ is simply a function that ensures 

The first two rules are fairly straight-forward. It says that it types $\true, \false$ as booleans. The latter two are what are of interest:
\begin{enumerate}
\item The third rule says that if $x$ if a linear endpoint in the input context, $x$ is typed as a linear endpoint. However, the output context does \textit{not} contain $x$. This is how we ensure that $x$ is used in a single process, since any other process that rely on $x$ will not contain $x$ in the context when typing the other processes.
\item This rule basically ensures that $x$ is defined in the context, and since the output context contains $x$ it must also ensure that $x$ isn't a linear endpoint, which is what $\unf{T}$ does.
\end{enumerate}

\sss{Typing rules for processes}
Now, we will give the typing rules associated with processes. Recall that we do not type processes directly. The purpose of the subsequent typing process is two-fold.

\begin{enumerate}

\item These typing rules will define under what conditions a process may exist, based on our input context and process we are typing. This allows us to recursively check each condition to validate the typing rule. For example, suppose we were checking $(\rest xy) P$. One condition of the associated typing rule involves extending the input context to contain the restriction and then check if $P$ is a valid process.  
\item These typing rules also define how a context evolves. For example, suppose we were type checking a parallel process $P \mid Q$. We will type check $P$ and $Q$ separately, but we ensure that the correct context is used to check $Q$ by evaluating $P$ first and using the ``remaining" context that it provides.
\end{enumerate}

To this end, the typing rules associated with processes are written as follows: $\Gamma_1 \vdash P : \Gamma_2 : L$, where $L$ is a set of variables. We read this as follows:
\begin{quote}
    If we type-check $P$ with context $\Gamma_1$, the remaining context is $\Gamma_2$ and $L$ contains all the variables consumed by $P$. The idea here is that $L$ is used in another typing rule to ensure that the variables in $L$ aren't used in that typing rule.
\end{quote}


As you encounter each typing rule (or pair of corresponding rules) we will provide an intuition behind the rule on how to read it, and the purpose of the rule, in terms of the behavior it is trying to enforce.
\todo{Change this a bit}

\sss{Typing rule(s) for $\ep$}
$$
\Gamma \vdash \ep ; \Gamma ; \emptyset
$$

Because processes are written recursively, naturally the typing rules are written recursively. As $\ep$ is the base process, our first typing rule involves typing $\ep$, as you might expect:

This rule is read as ``You can type $\ep$ with any context, with the remaining context being the input context, and it uses no linear variables".

\sss{Typing rule(s) for parallel processes}
$$
\frac{\Gamma_1 \vdash P : \Gamma_2 ; L_1 \quad \Gamma_2 \td L_1 \vdash Q : \Gamma_2 ; L_2}{\Gamma_1 \vdash P \mid Q : \Gamma_3 ; L_2}
$$

The idea behind this rule is that it types the two parallel processes independently, but it gets the set of linear variables used by $P$ and makes sure that $Q$ can be typed correctly without having that variable in the context.

\sss{Typing rule(s) for processes under restriction}
$$
\frac{\Gamma_1, x : T, y : \bar{T} \vdash P : \Gamma_2 ; L}{\Gamma_1 \vdash (\rest xy : T) P : \Gamma_2 \td \set{x,y} ; L \setminus \set{x,y}}
$$

The idea here is that to type a process under restriction, if we lift those channels endpoints to the greater context, it should be able to type the inner process. Observe that $(\rest xy : T)$. To improve the efficiency of the type-checking algorithm, we must supply the type of the restricted channel endpoints when we check our process. In principle, we should know this when we type our process, since we should be using our restricted channel inside $P$, and we type $T$ along how it's used in $P$.


\sss{Typing rule(s) for conditional processes}
$$
\frac{\Gamma_1 \vdash v : q\  \bool ; \Gamma_2 \quad \Gamma_2 \vdash P : \Gamma_3 ; L \quad \Gamma_2 \vdash Q : \Gamma_3 ; L}{\Gamma_1 \vdash \ifelsethen{v}{P}{Q} : \Gamma_3 ; L}
$$

The idea here is that we check that $v$ is indeed a boolean variable, and that $P$ and $Q$ are type-correct under the same context, since they will be run in the same context, after one is chosen to be run.


\sss{Typing rule(s) for value passing}

The next pair of rules define passing values along a channel

$$
\frac{\Gamma_1 \vdash x : q ! T . U ; \Gamma_2 \quad \Gamma_2 \vdash v : T ; \Gamma_3 \quad \Gamma_3 + x : U \vdash P : \Gamma_4 ; L}{\Gamma_1 \vdash \bar{x}v.P : \Gamma_4 ; L \cup (\ifelsethen{q = \lin}{\set{x}}{\emptyset})} \\
$$

To check outputting along a channel, we have three jobs (corresponding to the three pre-conditions in this typing rule):
\begin{enumerate}
    \item We have to make sure that the context expects $x$ to be used as an output endpoint. We do this directly, and pick up the continuation type in $U$ and the value type in $T$.
    \item That the value we are outputting ($v$) corresponds to the type that the context expects our output to be ($T$)
    \item That if we continue $x$ as the expected continuation type ($U$) that the context expects, then $P$ will still be typed correctly.
\end{enumerate}

Observe that we correctly get the linear variables used by process $P$ into $L$, which we add $x$ to if it is expected to be linear by the context.

$$
\frac{\Gamma_1 \vdash q_2?T.U ; \Gamma_2 \quad (\Gamma_2, y : T) + x : U \vdash P : \Gamma_3 ; L \quad q_1 = \un \implies L \setminus \set{y} = \emptyset}{\Gamma_1 \vdash q_1 x(y) . P : \Gamma_3 \td \set{y} ; L \setminus \set{y} \cup  (\ifelsethen{q_2 = \lin}{\set{x}}{\emptyset})}
$$

To check input along a channel, we again have three jobs:
\begin{enumerate}
    \item We have to make sure that the context expects $x$ to be used as an input endpoint. We do this directly, and pick up the continuation type in $U$ and the value type in $T$.
    \item If we add $y$ to the context with the expected type for it ($T$), $P$ is still well-typed. 
    \item If $x(y)$ is a replicating input (i.e: has $\un$ applied to it), then it cannot contain any linear variables, since the replication will result in the linear variable being used in multiple threads (the current thread and the thread it spawns). $L$ is the set of linear variables used by $P$, and we ensure that it is basically empty, except perhaps for $y$. We can use $y$ since $y$ is a ``local" variable in $P$.
\end{enumerate}

For our output, we ensure that $y$ is not present in the output context / set of linear variables, and that we add $x$ to the set of variables that shouldn't be used in another thread ($L$) if $x$ is a linear endpoint.

\sss{Typing rule(s) for branching}

The next pair of rules define branching / selection.

$$
\frac{\Gamma_1 \vdash x : q \& \set{l_i : T_i}_{i \in I} ; \Gamma_2 \quad \forall{i \in I}:\Gamma_2 + x : T_i \vdash P_i : \Gamma_3 ; L_i \quad \forall{i,j \in I} : L_i \setminus \set{x} = L_j \setminus \set{x}}{\Gamma_1 \vdash x \offer \set{l_i : P_i}_{i \in I} : \Gamma_3 ; L \cup (\ifelsethen{q = \lin}{\set{x}}{\emptyset})} \\
$$

To type-check a channel that offers multiple process pathways we need to:
\begin{enumerate}
    \item Check that the context expects $x$ to branch with the same labels as the process.
    \item That if the process branches on label $l_i$, then the process continues as type $T_i$.
    \item That each pathway uses the same linear variables, though they may differ with respect to using $x$. \todo{Why?}
\end{enumerate}

$$
\frac{\Gamma_1 \vdash x : q \oplus \set{l_i : T_i}_{i \in I} ; \Gamma_2 \quad \Gamma_2 + x : T_j \vdash P : \Gamma_3 ; L \quad j \in I}{\Gamma_1 \vdash x \select l_j.P : \Gamma_3 ; L \cup (\ifelsethen{q = \lin}{\set{x}}{\emptyset})}
$$

To type-check a channel that requests the other endpoint user to branch, we need to:
\begin{enumerate}
    \item Check that the context expects $x$ to request a branching operation, with the same labels
    \item That the subsequent process type-checks correctly, with the corresponding continuation type for $x$ based on the label selected.
    \item That the label selected is one that that the context expects for $x$. Note that it does \textbf{not} check that the label exists at the corresponding branching process.
\end{enumerate}

% \subsubsection{Table of rules}
\todo{pi table}