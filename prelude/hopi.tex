One extension of $\pi$-calculus, that can be applied to any process calculi really, is the idea of passing \textbf{abstractions}. Abstractions can be thought of as a mobile process, allowing one to instantiate a process based on a some configuration details that they choose. This allows for a more concise description of processes, but it also contributes to more ``real-world" efficiency, since the locally instantiated process can be thought of as "closer" and thus take less time communicating.

This specification of $\pi$-calculus Restrepo et al. \cite{main} is actually a specification for higher-order $\pi$-calculus, which is formally described in section \ref{hopisyn}

\todo{Figure of sorts: value $<$ channel $<$ abstraction}